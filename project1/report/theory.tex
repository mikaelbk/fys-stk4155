\newcommand{\colVec}[1]{\begin{bmatrix} #1_0 \\ #1_1 \\ \vdots \\ #1_{n-1} \\ #1_{n} \end{bmatrix}}

\section{Theory}
\subsection{Ordinary least squares}
We want to get a specific solution to the equation
\begin{equation}\label{eq: linearFit} \hat{y} = \hat{X}\hat{\beta}+\hat{\epsilon} \end{equation}
Where $\hat{y}$ is a vector of our measured values, $\hat{X}$ is a matrix containing variables and determines how we want to fit our data, $\hat{\beta}$ is a vector of the parameters for our fit and $\hat{\epsilon}$ is a vector representing the error in our datapoints (often termed the residuals, representing how far off our prediction is from the measurements). The variables $\hat{y}$ and $\hat{X}$ are fixed and we want to choose parameters $\hat{\beta}$ in such a way that the errors $\hat{\epsilon}$ are minimized. An example might help clarify the situation\\\\

Lets say we have conducted an experiment where we have measured the position of a ball launched straight up into the air from a cannon. Neglecting air resistance we know that the analytical solution is on the form of a second order polynomial $x(t)=x_0+v_0t+at^2$, where $x_0$ and $v_0$ are the initial conditions for the position and velocity respectively. This analytical solution is our model, but the actual measured values could (and indeed probably would) differ from this, simply due to errors in the measurement or other effects coming into play that the model has not accounted for (like air resistance). In any case, if we measured the position $n$ times our linear algebra problem could be stated like this:
\begin{align*}
	\begin{bmatrix}
		x(t_0) \\ x(t_1) \\ \vdots \\ x(t_{n-1}) \\ x(t_n)
	\end{bmatrix}
	=
	\begin{bmatrix}
		(t_0)^0	&	(t_0)^1	&	(t_0)^2 \\
		(t_1)^0	&	(t_1)^1	&	(t_1)^2 \\
		\vdots	&	\vdots	&	\vdots \\
		(t_{n-1})^0	&	(t_{n-1})^1	&	(t_{n-1})^2 \\
		(t_{n})^0	&	(t_{n})^1	&	(t_{n})^2
	\end{bmatrix}
	\begin{bmatrix}
		x_0 \\ v_0 \\ a
	\end{bmatrix}
	+
	\colVec{\epsilon}
\end{align*}
Or, again, stated through vectors and matrix notation
\begin{align*}
	\hat{x} = \hat{T} \hat{\beta} + \hat{\epsilon}
\end{align*}
With some of the variable names adjusted simply to better indicate the represented values we have in this problem. Essentially we want to determine the variables $x_0$, $v_0$ and $a$ so that the error terms are minimized (the equation has solutions for all $\hat{\beta}$s, but most of those are horrible fits with huge error terms). Note that there are two dimensionalities coming into play here. $n$ is the number of measurements and determines the length of the column vectors $\hat{x}$ and $\hat{\epsilon}$. In addition there is a second dimension that determines the number of columns in the matrix $\hat{T}$ and the length of the vector $\hat{\beta}$. This number, say $m$, indicates the complexity of our model. When doing a polynomial fit, $m-1$ is the order of the polynomial we want to fit our data to. So in this case, where we want to fit a second degree polynomial, we have $m=3$.\\\\

The example above gives an impression of what the variables in the linear algebra equation represents but is quite specific and we can generalize a bit. For instance, it is not necessary to fit a polynomial at all. By changing our $\hat{X}$ matrix we could fit to any orthogonal function that we would like. The number of datapoints can be anything we want as log as $m\leq n$.\\

A key part of the fitting is how to minimize the so called "cost-function". In our case we want to minimize the $\epsilon$'s. There are different ways of doing this, but perhaps the simplest one is to do an \textbf{Ordinary Least Squares} (OLS) fit. That is to say when taking the difference between our predicted values and the measured values (essentially being the residuals) we want the squared sum of the difference for each datapoint to be as low as possible. That is to say that we want to minimize the function
\begin{align*}
	Q = \sum_{i=0}^{n-1}\epsilon_i^2 = \hat{\epsilon}^T\hat{\varepsilon} = \qty(\hat{y}-\hat{X}\hat{\beta})^T\qty(\hat{y}-\hat{X}\hat{\beta})
\end{align*}
We can see that taking the squared sum of all the elements in $\hat{\epsilon}$ is the same as taking the inner product of the vector with itself, and that we can use equation \ref{eq:fit} to develop the expression further. We are interested in finding the parameters $\hat{\beta}$ that leads to the minimization of the squared sum of the residuals. So now that we have a function of the squared sum of the residuals with $\hat{\beta}$ as a variable it is a simple matter of finding when the derivative of this function with respect to $\hat{\beta}$ is zero; as this will give us the minimum (it must, of course, be a minimum as only a quite specific $\hat{\beta}$ will give a good fit and other values, deviating from this, would only increase the residuals squared sum. The possibility of more than one minimum is not something we need to worry about at this point). We will here state the derivative without further explanation, suffice it to say that looking at the expanded indexed expression (i.e. not on vector form) one quickly comes to the conclusion that this must indeed be the correct derivative.
\begin{align*}
	\pdv{Q(\hat{\beta}}{\hat{\beta}} = 0 = \hat{X}^T\qty(\hat{y}-\hat{X}\hat{\beta})\\
	\hat{X}^T\hat{y} = \hat{X}^T\hat{X}\beta\\
	\hat{\beta} = \qty(\hat{X}^T\hat{X})^{-1}\hat{X}^T\hat{y}
\end{align*}
And by the magic of linear algebra we have arrived at an analytical solution for the parameters $\hat{\beta}$ we need for an OLS regression. It is, however, important to note that we assume that the matrix $\hat{X}$ here is invertible.

\subsubsection{Mean square error}
The Mean Square Error (\textbf{MSE}) can give a measure of the quality of our estimator. It is defined as
\begin{equation}\label{eq: mse}
	MSE(\epsilon) = \frac{1}{n}\sum_n^{n-1}\epsilon^2
\end{equation}
As such it can be thought of  as the average of the square of our residuals. We see that our OLS method minimizes the MSE of our predictor variables. Of course, as easily seen from the definition, the MSE can never be negative and lower values means that we have a better prediction (at zero there is a perfect fit).

%\subsubsection{The chi squared function $\chi^2$}
%\begin{equation}\label{chi}
%	\chi^2 = \sum_{i=0}^{n-1}\qty(\frac{\epsilon_i}{\sigma_i})^2 = \hat{\epsilon}^T \hat{\Sigma}^{-2}\hat{\epsilon}
%\end{equation}

\subsection{Coefficient of variation - $R^2$ score}
In regression validation the $R^2$ is the gold standard when it comes to measuring goodness of fit. In straight terms it is the proportion of the variance in the dependent variable that is predictable from the independent variable(S) [source wikipedia - coefficient of determination]. 
\begin{equation}\label{eq: R squared}
	R^2 = 1 - \frac{ \sum(y_i-\tilde{y}_i)^2 }{ \sum(y_i-\bar{y}_i)^2 }
\end{equation}
Where $y_i$ are the indexed response variables (data we want to fit) and $\tilde{y}_i$ is the predictor variables from our model (so $\epsilon_i = y_i - \tilde{y}_i$). The average of the response variables is denoted $\bar{y}_i$. In our case it the second term can also be considered as the ratio of \textbf{MSE} to the variance (the $1/n$ factors kill each other in a fraction). Let's interpret the formula step by step to get an impression of how it differentiates a poor from a good fit. If the residual sum of squares (SS\textsubscript{res}) is low we have a good fit. However, we should compare this to the spread of our response variables. After all, if the response variables are all nicely distributed close to the mean then getting a good SS\textsubscript{res} is not that impressive. We therefore do a sort of normalization in the fraction, taking the scale of our data into consideration. In the simplest polynomial fit, using a zeroth order polynomial (just a constant), we see that our model would just be a constant function of the mean. The sums would be equal, returning unity on the fraction and the total $R^2$ score would be zero. In the other extreme, if our model fits perfectly, then SS\textsubscript{res} would be zero and the $R^2$ score would be one. In this sense we have a span of possible $R^2$ scores between zero and one, from the baseline of the simplest model at zero and a perfect fit at one. 
\subsection{Ridge}
\subsection{Lasso}
\subsection{K-fold and and bootstrap}
% three central statistical quantities: variance, covariance and expectation value
