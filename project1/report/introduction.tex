\section{introduction}
A typical problem within the natural sciences is how to interpret the trends and behavior of the results and data from an experiment. As a first approximation this will often be done qualitatively, e.g. "These values appear to increase linearly with time", but a more rigorous approach through regression analysis and resampling methods is eventually more preferable. The theory and implementation of this regression analysis assignment will also provide valuable knowledge relevant for machine learning like minimization of the cost function and dealing with underfitting and overfitting.\\
Initially we consider Franke's function (a standard two-dimensional function to use in combination with interpolation and fitting algorithms [oppgavetekst]) as we try to fit different ordered polynomials to it using regression methods ranging from the classical Ordinary Least Squares (\textbf{OLS}) to Ridge and Lasso. This is all implemented through python where we make sure to also utilize the scipy package with built in tools for regression in order to compare our results. In addition we use the k-fold resampling technique to get an estimate of the predictive power of the different models and adjust certain parameters. To finalize this first part we compare the numerical requirements of the models and how the accuracy vs computational intensity varies between them. The OLS is employed through an analytical linear algebra expression involving a matrix inversion explained in the theory. The Ridge and Lasso regressions operate in much the same way, although an analytical expression does not exist for the latter (this has no other immediate consequences than somewhat complicating the programming technicalities).\\
In the second part of the assignment we consider actual sampled terrain data. The techniques explored in the first part are then applied in a similar fashion to the real data, illustrating how they perform on a more natural dataset.